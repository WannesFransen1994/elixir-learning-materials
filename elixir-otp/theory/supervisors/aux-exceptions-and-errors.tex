

\section{Exceptions \& Errors}

\frame{\tableofcontents[currentsection]}

\begin{frame}
    \frametitle{Fundamentals of an exception}
    What is an exception?
    \begin{itemize}
        \item Exceptional behaviour that is abnormal
        \item Causes the current operation / process to crash
    \end{itemize}

    \vfill

    Examples
    \begin{itemize}
        \item HTTP request
        \item DB: Record locking (during insert)
    \end{itemize}
\end{frame}

\begin{frame}
    \frametitle{Programming styles to counter this}
    \begin{itemize}
        \item Total programming (program for all scenarios)
        \item Nominal programming (uses preconditions - cannot run outside nominal conditions)
        \item Defensive programming (use of exceptions)
    \end{itemize}
\end{frame}

\begin{frame}
    \frametitle{Let it crash (offensive programming)}
    \begin{itemize}
        \item Fail fast
        \item Only write code for the happy path
        \item Failures are completely isolated
        \item With try/catch, what about abnormal states?
        \item Restarting is often quicker than huge try/catch blocks
        \item Not always applicable
    \end{itemize}

    \vfill

    \tiny
    Excellent
    \href{https://elixirforum.com/t/understanding-the-advantages-of-let-it-crash-term/9748/19?u=wfransen}{forum post}
    from Sasa Juric and a
    \href{https://elixirforum.com/t/understanding-the-advantages-of-let-it-crash-term/9748/21?u=wfransen}{video}
    from Joe Armstrong
\end{frame}
