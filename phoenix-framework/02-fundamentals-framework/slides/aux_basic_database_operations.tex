\section{Getting started with Ecto}

\frame{\tableofcontents[currentsection]}

\begin{frame}
    \frametitle{Ecto - the concepts}

    \begin{itemize}
        \item Repo module - Via the repository, we can create, update, destroy and query existing entries.
        \item Schemas - used  to map data sources to Elixir structs.
        \item Changesets - way to filter and cast external parameters, as well to validate changes before applying them
        \item Query - queries written in Elixir syntax with specific DSL. Queries are by default secure, avoiding common problems. These can be created composable / piece by piece
    \end{itemize}
\end{frame}

\begin{frame}
    \frametitle{Ecto SQL - not the same thing!}

    \begin{itemize}
        \item This provides functionality for working with SQL databases in Ecto
        \item Migrations are an example of this
    \end{itemize}
\end{frame}

\begin{frame}[fragile]
    \frametitle{There's already a great guide for this}

    \begin{itemize}
        \item Database is up to you (MySQL might be the easiest)
        \item Feel free to use generators such as:
        \begin{verbatim}
mix phx.gen.schema User users \ 
name:string email:string \
bio:string number_of_pets:integer
        \end{verbatim}
    \end{itemize}

    \vfill

    \href{https://hexdocs.pm/phoenix/ecto.html\#content}{[LINK]}
\end{frame}