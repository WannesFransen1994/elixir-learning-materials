\section{Getting started with Ecto}

\frame{\tableofcontents[currentsection]}

\begin{frame}
    \frametitle{Ecto - the concepts}

    \begin{itemize}
        \item \href{https://hexdocs.pm/phoenix/ecto.html\#repo-configuration}{Repo module} - Via the repository, we can create, update, destroy and query existing entries.
        \item \href{https://hexdocs.pm/phoenix/ecto.html\#the-schema}{Schemas} - used  to map data sources to Elixir structs.
        \item \href{https://hexdocs.pm/phoenix/ecto.html\#changesets-and-validations}{Changesets} - way to filter and cast external parameters, as well to validate changes before applying them
        \item Query - queries written in Elixir syntax with specific DSL. Queries are by default secure, avoiding common problems. These can be created composable / piece by piece
    \end{itemize}
\end{frame}

\begin{frame}
    \frametitle{Ecto SQL - not the same thing!}

    \begin{itemize}
        \item This provides functionality for working with SQL databases in Ecto
        \item Migrations are an example of this
    \end{itemize}
\end{frame}

\begin{frame}[fragile]
    \frametitle{\href{https://hexdocs.pm/phoenix/ecto.html\#content}{Setup}}

    \begin{itemize}
        \item Database is up to you
        \item Configure your database: \\
         development: In /config/dev.exs \\
         production: As environment variables defined in /config/prod.secret.exs
        \item Use generators such as:
        \begin{verbatim} mix phx.gen.schema User users \ 
 name:string email:string \
        \end{verbatim}
        \item migrate database:
        \begin{verbatim} mix ecto.migrate \end{verbatim}
    \end{itemize}
\end{frame}